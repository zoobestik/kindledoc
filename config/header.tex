\usepackage{xcolor}
\usepackage{listings}
\usepackage{fancyvrb}

\definecolor{codebg}{HTML}{F7F7F7} % choose any hex color you like
\definecolor{shadecolor}{HTML}{F7F7F7} % background for code blocks

\lstset{
  backgroundcolor=\color{codebg},
  % optional niceties:
  frame=single,
  rulecolor=\color{black},
  framesep=6pt
}

% Inline code background without soul/fancyvrb
\makeatletter
\let\OldTexttt\texttt
\newcommand{\InlineCode}[1]{
  \begingroup
  \setlength{\fboxsep}{1.5pt}% padding around inline code
  \colorbox{codebg}{\OldTexttt{#1}}
  \endgroup
}
\renewcommand{\texttt}[1]{\InlineCode{#1}}
\makeatother

\usepackage[normalem]{ulem} % provides \uline without changing \emph

% Underline the visible text of \href
\makeatletter
\let\OldHref\href
\renewcommand{\href}[2]{\OldHref{#1}{\uline{#2}}}
\makeatother

% Underline bare URLs (\url). Use url’s DeclareUrlCommand so verbatim handling stays intact.
\DeclareUrlCommand\ULurl{
  \def\UrlFont{\ttfamily}
  \urlstyle{same}
  \uline
}
\let\OldUrl\url
\renewcommand{\url}[1]{\ULurl{#1}}

% Emoji and symbol fallback for LuaLaTeX:
% Use HarfBuzz renderer and fall back to Noto Color Emoji, then DejaVu Sans.
\usepackage{fontspec}
\defaultfontfeatures{Renderer=Harfbuzz}

\directlua{
  luaotfload.add_fallback("emoji_fallback", {
    "Apple Color Emoji:mode=harf;",
    "Noto Color Emoji:mode=harf;",
    "DejaVu Sans:mode=harf;"
  })
}

% Re-apply fonts with fallback so missing glyphs (e.g., emoji) render automatically.
\setmainfont{Amazon Ember}[RawFeature={fallback=emoji_fallback}]
\setsansfont{Amazon Ember}[RawFeature={fallback=emoji_fallback}]
\setmonofont{Fira Code}[RawFeature={fallback=emoji_fallback}]
